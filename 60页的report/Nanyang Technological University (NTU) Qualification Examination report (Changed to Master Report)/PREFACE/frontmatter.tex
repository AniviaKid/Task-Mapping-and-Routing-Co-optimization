% ---------------------------------------------------------------------------- %
% Front Matters
% ---------------------------------------------------------------------------- %
% ---------------------------------------------------------------------------- %
% Acknowledgement
% ---------------------------------------------------------------------------- %
% ---------------------------------------------------------------------------- %
% Abstract
% ---------------------------------------------------------------------------- %

\setsinglespace
\begin{nabstract}
% ----------------------------------------------------------------------------
% Abstract
% ----------------------------------------------------------------------------
\indent In the internet-of-things (IoT) era, the edge computing encourages the development of application-specific integrated circuit (ASIC) based accelerators. However, based on the modified Amdahl's law which considers the effect of communication and synchronization in multi-core systems \cite{yavits2014effect}, the communication bottleneck damps the speedup gained by parallelism and computation acceleration. To connect different processing elements (PE) in the accelerator, network-on-chip (NoC) is generally deployed. Nevertheless, considering the performance demand (i.e., latency and throughput) and cost budget (i.e., power and area), the conventional NoCs with hop-by-hop traversal cannot achieve these two targets simultaneously for ASIC. 
The performance and cost of traditional NoC are limited by two issues. One is the hop-by-hop arbitration and buffering result in time- and power-consuming communication, especially for the long distance data transmission.
The other is network contention among packets with shared links which can result in huge delay and drastically degrade the NoC performance, especially for communication-intensive applications.
To address these two issues, we propose a novel NoC architecture, CDless NoC. Specifically, we design the configurable router together with cluster-based controllers to replace the distributed per-hop arbitration. Also, we propose matched routing algorithms for the controller with the consideration of the run-time network state. With the help of our proposed hardware/software co-design, the single-cycle long-distance communication path can be efficiently established, which fully boost the system performance efficiency.

\newpage
\end{nabstract}

\setonehalfspace
\addcontentsline{toc}{section}{Table of Contents}
\tableofcontents
\listoffigures
\listoftables
\begin{table}
    \caption{Notations Used in This Paper}
    \renewcommand{\arraystretch}{1.0}
    \arrayrulewidth=0.85pt
    \tabcolsep 3pt
    \centering
    \footnotesize
    \begin{tabular}[c]{c  c}
        \hline
        \hline
        {\bf Notation }&\bf{Description}\\
        \hline
        $\mathcal{V}, \mathcal{E}$ &  The set of tasks nodes and edges. \\
        $m,p,l$ & The set of message, packet and flit. \\
        $|.|$ & The number of elements.\\
        $\mathcal{A}$& The set of processing elements. \\
        $\tau_{v,a}$& $=q_v/ep_a$. The execution time of $v$ on $a$. \\
        $\mathcal{F},\mathcal{G}$& The mapping and routing algorithm. \\
        $\gamma_m$& The route to transmit data of message $m$. \\
        ${L}$ & The latency related to transmission. \\
        % ${L}_{head},{L}_{tail}$ & The latency for head and tail filts.\\
        $E$ & The energy used for transmission.\\
        $t_r, t_w$ & The router-stage latency and propagation delay.\\
        $\mathcal{M},\mathcal{R}$& The mapping and routing result \\
        $\mathcal{M}(t)$& Processing element mapped to task $t$ \\
        \hline
        
    \hline
    \end{tabular}
    \label{notations}

\end{table}

\newpage
\startarabicpagination
%END
